\section{Nutzeranleitung}
\subsection{Nutzung}
NUTZUNG HIER

\subsection{Deployment}
Um den \vires-Server zu starten, muss lediglich die .zip-Datei für das jeweilige Betriebssystem und die jeweilige Prozessorarchitektur unter 
\url{BUILD_LINK_HIER_EINFÜGEN} heruntergeladen, entpackt und über die Datei \verb+vires+ (bzw. unter Windows \verb+vires.exe+) ausgeführt werden.

\subsection{Building}
Bei der Kompilierung des Projekts wird zwischen Backend und Frontend unterschieden:
\begin{itemize}
	\item Für die Kompilierung des Backends wird zuerst eine Go-Installation benötigt.\\
	Hierfür muss eine aktuelle Go-Version von \url{https://golang.org/dl/} heruntergeladen und hieraufhin ein Go-Workspace eingerichtet werden.\\
	Zur Einrichtung eines Go-Workspaces muss ein Ordner für den Workspace erstellt werden und hieraufhin der Pfad des Ordners in der Umgebungsvariable 
	\verb+$GOPATH+ gesetzt werden.\\
	Nach der Einrichtung des Workspaces können der \vires-Sourcecode und alle Abhängigkeiten mithilfe des Konsolenkommandos
	\verb+go get github.com/mhuisi/vires/...+ heruntergeladen und für das aktuelle Betriebssystem und die aktuelle Architektur kompiliert werden. 
	Die reine Binärdatei kann dann in \verb+$GOPATH/bin+ aufgefunden werden und sollte hieraufhin nach
	\verb+$GOPATH/src/github.com/mhuisi/vires/src/vires+ bewegt werden.
	\item Da das Frontend vom Server bereitgestellt werden soll, befinden sich alle Dateien des Frontends in einem Unterordner des Backends. Es ist deshalb erforderlich zuerst die Einrichtung des Workspaces - wie im Buildprozess des Backends beschrieben - durchzuführen.
	Um das Frontend kompilieren zu können, muss zunächst Coffee Script auf dem Rechner eingerichtet werden. Hierzu muss zunächst eine aktuelle Version von NodeJS (\url{https://nodejs.org/en/}) istalliert werden. Sobald NodeJS korrekt eingerichtet ist, kann der Coffee Compiler mit dem Kommando \verb+npm install -g coffee-script+ installiert werden. Alternativ finden sich unter \url{http://coffeescript.org/\#installation} weitere optionen zur Installation von Coffee Script.
	Um die Dateien des Frontends zu kompilieren, muss in der Konsole in den Ordner \verb+$GOPATH/src/github.com/mhuisi/vires/src/vires/res+ gewechselt werden. Dort kann dann das Kommando \verb+coffee -c -b -o js/ src/+ ausgeführt werden, um alle \verb+.coffee+ Dateien aus \verb+src+ nach \verb+js+ zu kompilieren.
\end{itemize}