\section{Nutzeranleitung}
\subsection{Nutzung}
NUTZUNG HIER

\subsection{Deployment}
Um den \vires-Server zu starten, muss lediglich die .zip-Datei für das jeweilige Betriebssystem und die jeweilige Prozessorarchitektur unter 
\url{BUILD_LINK_HIER_EINFÜGEN} heruntergeladen, entpackt und über die Datei \verb+vires+ (bzw. unter Windows \verb+vires.exe+) ausgeführt werden.

\subsection{Building}
Bei der Kompilierung des Projekts wird zwischen Backend und Frontend unterschieden:
\begin{itemize}
	\item Für die Kompilierung des Backends wird zuerst eine Go-Installation benötigt.\\
	Hierfür muss eine aktuelle Go-Version von \url{https://golang.org/dl/} heruntergeladen und hieraufhin ein Go-Workspace eingerichtet werden.\\
	Zur Einrichtung eines Go-Workspaces muss ein Ordner für den Workspace erstellt werden und hieraufhin der Pfad des Ordners in der Umgebungsvariable 
	\verb+$GOPATH+ gesetzt werden.\\
	Nach der Einrichtung des Workspaces können der \vires-Sourcecode und alle Abhängigkeiten mithilfe des Konsolenkommandos
	\verb+go get github.com/mhuisi/vires/...+ heruntergeladen und für das aktuelle Betriebssystem und die aktuelle Architektur kompiliert werden. 
	Die reine Binärdatei kann dann in \verb+$GOPATH/bin+ aufgefunden werden und sollte hieraufhin nach
	\verb+$GOPATH/src/github.com/mhuisi/vires/src/vires+ bewegt werden.
	\item FRONTEND-BUILD-PROZESS HIER EINFÜGEN
\end{itemize}