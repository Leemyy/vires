\section{Projektreflexion}
Das Ergebnis des Projektes ist für uns zufriedenstellend, da wir bei der Entwicklung des Projekts eine Menge neuer Technologien angewandt und mehrere eigene Algorithmen entwickelt haben. Was den Lerneffekt angeht war dieses Projekt ein voller Erfolg - algorithmisch war dieses Projekt sehr anspruchsvoll. \\
Hiermit hängt auch zusammen, weshalb die ursprünglichen Ziele bei weitem nicht erreicht wurden: Die Komplexität vieler der zu lösenden Probleme wurde massiv unterschätzt. Das Erlernen von Go, den Konzepten von CSP, Coffeescript, Javascript, WebGL, verschiedenen Libraries, das eigene Rendern des Frontends, die Verwendung einer Timer-Architektur, die Verwendung einer Priori-Kollisionsdetektion, das Schreiben eines eigenen Schedulers und die Generation von zufälligen Maps mithilfe eines genetischen Algorithmus stellen insgesamt in der Summe eine große Aufgabe mit vielen zu lösenden Problem dar, für die es nicht immer Literatur gibt, auf die man zurückgreifen kann. \\
Die Aufgaben, welche noch nicht erfüllt wurden, weisen jedoch im Vergleich zu den erfüllten Aufgaben einen minimalen Lerneffekt auf. Insofern ist das Fehlen dieser Funktionen zwar wichtig für das Programm selbst, nicht jedoch für den Lerneffekt. \\
Das Projektmanagement des Projektes hätte deutlich besser laufen können - wäre kontinuierlich an dem Projekt gearbeitet worden, so wäre das gesamte Projekt komplett ohne Probleme fertiggestellt worden. \\
Insgesamt war das Projektmanagement mangelhaft, das Ergebnis unvollständig, die Aufgabe überfordernt - und trotzdem das Projekt ein voller Erfolg, da es viele Möglichkeiten geschaffen hat, sich mit neuen Technologien auseinanderzusetzen und komplizierte algorithmische Probleme zu lösen.