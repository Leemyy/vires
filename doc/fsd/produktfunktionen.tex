\section{Produktfunktionen}

% An \idit with an F as prefix.
% The first argument is the ID 
% of the function and the
% second argument is the name of
% the function.
\newcommand{\fn}[2]{\idit{F#1}{#2}}

\subsection{Benutzerfunktionen}
\begin{description}
	\fn{0010}{Eingabe des Benutzernamens} 
		Jeder Nutzer hat einen Nutzernamen, was es einfacher machen soll, Cells und Vires bestimmten Namen zuzuordnen.
		Der Nutzername kann nach dem Betreten der Hauptseite eingegeben werden.
		\begin{itemize}
			\item Nutzernamen müssen nicht eindeutig sein: Mehrere Nutzer können den gleichen Namen verwenden. 
			\item Jeder Nutzername ist auf 20 Zeichen limitiert.
			\item Wird der Nutzername nicht angegeben, so gilt dieser automatisch als der Nutzername ``Unknown''.
			\item Nutzernamen werden über Cookies bei den Nutzern abgespeichert.
			\item Besuchen Nutzer die Hauptseite erneut, so befinden sich ihre Nutzernamen bereits im Eingabefeld.
		\end{itemize}
	\fn{0020}{Matchmaking} 
		Auf der Hauptseite können Nutzer dem Matchmaking beitreten. Das Matchmaking sorgt dafür, dass Nutzer in einen guten Room weitergeleitet werden. \\
		Ein guter Room ist wie folgt definiert:
		\begin{itemize}
			\item Der Room befindet sich aktuell in der Pre-Match-Phase.
			\item Die Maximalanzahl an Players in dem Room ist noch nicht erreicht.
			\item Der Room hat die geringste Anzahl an Players aus allen Rooms, die in Frage kommen.
		\end{itemize}
		Lässt sich kein guter Room finden, so wird ein neuer Room erstellt, in den die Nutzer dann weitergeleitet werden.
	\fn{0030}{Direktes Beitreten}
		Mithilfe der Room-ID können Nutzer auf der Hauptseite einem bestimmten Room direkt beitreten. Ist die Maximalanzahl an Players des Rooms erreicht, so treten Nutzer dem Room automatisch als Spectators bei.
	\fn{0040}{Phases}
		Jeder Room befindet sich entweder in der Pre-Match-Phase oder in der Match-Phase. Wird die eine Phase beendet, so wird direkt die nächste Phase gestartet.
	\fn{0050}{Pre-Match-Phase}
		Während der Pre-Match-Phase wird auf Players für die Match-Phase gewartet. 
	\fn{0060}{Countdown}
		Sind zwei oder mehr Players im Room, so fängt ein Timer an, von 60 Sekunden aus herunterzuzählen. Erreicht der Timer null Sekunden, so beginnt das Match. Während dieser Wartezeit können die Players abstimmen, ob sie das Match sofort starten wollen. Die Abstimmung dauert 10 Sekunden und kann nur mit einer Zweidrittelmehrheit der an der Abstimmung teilnehmenden Players entschieden werden.
	\fn{0070}{Match-Phase}
		Während der Match-Phase wird von den Players \vires\ gespielt, während Spectators das Spiel beobachten.
	\fn{0080}{Rollen}
		Tritt man einem Room bei, so ist man standardmäßig Player. Players können sich in einem Room dazu entscheiden, zu Spectators zu werden.
		\begin{itemize}
			\item Lief während der Entscheidung der Timer, existiert aber nach der Entscheidung nur noch ein Player, so wird der Timer abgebrochen.
			\item War der Player an einer Abstimmung beteiligt, so wird seine Stimme zurückgezogen. 
			\item War der Player aktuell in der Match-Phase aktiv, so wird eine Neutralization auf alle seine Manned Cells durchgeführt und seine Moving Vires entfernt.
		\end{itemize}
		Während der Pre-Match-Phase können Spectators zurück zu Players wechseln, insofern die Maximalanzahl an Players des Rooms nicht erreicht ist. Das Zurückwechseln is während der Match-Phase nicht möglich.
	\fn{0090}{Maximalanzahl an Players}
		Jeder Room hat eine Maximalanzahl von 20 Players. Rooms können insgesamt auch mehr Nutzer enthalten, insofern die restlichen Nutzer alle Spectators sind.
	\fn{0100}{Room-ID}
		Jeder Room hat eine einzigartige Room-ID, welche sich auch in der URL des Rooms wiederfinden lässt. Die Room-ID kann verwendet werden, um einem Room direkt beizutreten.
\end{description}

\subsection{Administratorfunktionen}