\section{Produktfunktionen}

% An \idit with an F as prefix.
% The first argument is the ID 
% of the function and the
% second argument is the name of
% the function.
\newcommand{\fn}[2]{\idit{F#1}{#2}}

\subsection{Benutzerfunktionen}
\subsubsection{Hauptseite}
Betritt der Nutzer die Hauptseite der Website, so kann er seinen Nutzernamen eingeben, dem Matchmaking beitreten oder direkt einem Match über die Room-ID des Matches beitreten.

\begin{description}
	\fn{0010}{Eingabe des Benutzernamens} 
		Jeder Nutzer hat einen Nutzernamen, was es einfacher machen soll, die Farben eines Players einem Namen zuzordnen. 
		\begin{itemize}
			\item Nutzernamen müssen nicht eindeutig sein: Mehrere Nutzer können den gleichen Namen verwenden. 
			\item Jeder Nutzername ist auf 20 Zeichen limitiert.
			\item Wird der Nutzername nicht angegeben, so gilt dieser automatisch als der Nutzername ``Unknown''.
			\item Nutzernamen werden über Cookies beim Nutzer abgespeichert.
			\item Besucht ein Nutzer die Hauptseite erneut, so befindet sich sein Nutzername bereits im Eingabefeld.
		\end{itemize}
	\fn{0020}{Matchmaking} 
		Auf der Hauptseite kann der Nutzer dem Matchmaking beitreten. Das Matchmaking sorgt dafür, dass der Nutzer an einen guten Room weitergeleitet wird. \\
		Ein guter Room ist wie folgt definiert:
		\begin{itemize}
			\item Der Room befindet sich aktuell in der Pre-Match-Phase.
			\item Die Maximalanzahl an Players in dem Room ist noch nicht erreicht.
			\item Der Room hat die geringste Anzahl an Players aus allen Rooms, die in Frage kommen.
		\end{itemize}
		Lässt sich kein guter Room finden, so wird ein neuer Room erstellt, in den der Nutzer dann weitergeleitet wird.
	\fn{0030}{Direktes Beitreten}
		Mithilfe der Room-ID kann der Nutzer auf der Hauptseite einem bestimmten Room direkt beitreten. Wird keine Room-ID angegeben, so wird der Nutzer stattdessen ins Matchmaking weitergeleitet.
\end{description}