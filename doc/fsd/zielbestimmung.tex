\section{Zielbestimmung}

Im Folgenden werden eine Menge spielspezifische Begriffe verwendet, welche im Glossar (unter \ref{sec:glossar}) nachgeschlagen werden können.

\subsection{Musskriterien}
\begin{description}
	\crit{Nutzer}{
		\item Nach dem Besuchen der Hauptseite der Website können Nutzer ihren Nutzernamen eingeben, dem Matchmaking beitreten oder direkt einem Match über die Room-ID beitreten.
		\item In der Pre-Match-Phase können Nutzer auswählen, ob sie Player oder Spectator sein wollen.
		\item Nutzer können einem Room über die URL mit der Room-ID des Raumes beitreten.
	}
	\crit{Player}{
		\item Während der Pre-Match-Phase können Players dafür abstimmen, dass das Match sofort gestartet wird.
		\item Während der Match-Phase können Players \vires\ (unter \ref{sec:spielerklaerung} erklärt) spielen.
		\item Zu jeder Zeit können sich Players dazu entscheiden, zum Spectator zu werden.
	}
	\crit{Spectator}{
		\item Während der Match-Phase können Spectators das Match beobachten.
		\item Während der Pre-Match-Phase können sich Spectators dazu entscheiden, zum Player zu werden.
	}
	\crit{Administrator}{
		\item Administratoren können das client- und das serverseitige Fehler-Log einsehen.
		\item Administratoren können das Spiel während der Laufzeit konfigurieren.
	}
\end{description}

\subsection{Wunschkriterien}
\begin{description}
	\crit{Nutzer}{
		\item Nutzer können in einem Room mit anderen Nutzern chatten.
		\item Nutzer können nach dem Besuchen der Website eine Theme auswählen.
	}
	\crit{Administrator}{
		\item Administratoren können das Spiel über eine Administrationskonsole verwalten.
	}
\end{description}

\subsection{Abgrenzungskriterien}
\begin{itemize}
	\item Das Spiel soll nur auf Englisch verfügbar sein.
	\item Spieler sollen nicht in der Lage sein, Spielelemente sichtbar für andere Spieler anzupassen.
	\item Spieler sollen im Chat nicht in der Lage sein, private Nachrichten zu schreiben.
	\item Das spielunterstützte Bilden von Teams soll nicht möglich sein.
	\item Das Spiel soll keine Daten über die Nutzung der Applikation sammeln.
\end{itemize}