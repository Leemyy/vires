\section{Spielerklärung}
\label{sec:spielerklaerung}

\subsection{Das Spiel}
Bei ``\vires'' handelt es sich insgesamt um ein einfaches RTS-Spiel. 
Das Spiel wird in einer Draufsicht gespielt.

Auf dem Spielfeld befinden sich mehrere Cells, welche im Spiel als Kreise dargestellt werden.
Cells produzieren Vires, welche die eigentlichen Einheiten des Spiels darstellen. Dieser Produktionsvorgang wird als Replication bezeichnet. \\
Jede Cell hat eine gewisse Cell-Force, welche durch die Größe der Cell darstellt wird.
Innerhalb eines gewissen Zeitintervalls, welcher von der Cell-Force abhängt, erhöht sich die Menge an Stationed Vires, also die Menge an Vires in einer Cell, um ein Virus.
Hat die Cell eine geringe Cell-Force, so ist auch die Replication dieser Cell langsamer. \\
Jede Cell hat eine Capacity, also einen gewissen Maximalumfang an Vires, den sie enthalten kann.
Die Capacity wird ebenfalls von der Cell-Force bestimmt: Je geringer die Cell-Force, desto kleiner ist auch die Capacity. \\
Jeder Player, also jeder aktiv am Match teilnehmende Nutzer, hat am Anfang des Matches eine Start-Cell. 
Alle Start-Cells auf dem Feld haben die gleiche Cell-Force. \\
Alle Cells auf dem Feld, die einem Player gehören, werden auch als Manned Cells bezeichnet.
Alle Cells auf dem Feld, die keinem Player gehören, sind Neutral Cells. Neutral Cells verhalten sich wie Manned Cells, haben jedoch eine langsamere Replication.

Vires können von Players, ausgehend von Manned Cells, die ihnen gehören, zu beliebigen anderen Cells bewegt werden. 
Jedes Movement, also jede Bewegung von Vires, bewegt die Hälfte der Stationed Vires in der jeweiligen Manned Cell in Richtung der Target-Cell.
Es können auch mehrere Movements gleichzeitig auf die gleiche Target-Cell gestartet werden, indem Players mehrere ihrer Manned Cells auswählen. \\
Die Vires innerhalb eines Movements heißen Moving Vires: Je größer die Anzahl an Moving Vires innerhalb eines Movements, desto größer ist die Movement-Force.
Je kleiner die Movement-Force eines Movements ist, desto schneller ist auch das Movement.
Movements werden ebenfalls durch Kreise dargestellt, deren Größe von der Movement-Force bestimmt wird. \\
Treffen mehrere Movements des gleichen Players auf die gleiche Target-Cell, so vereinen sich die Movements und bilden ein einzelnes Movement mit der vereinten Movement-Force der beiden Movements.
Treffen die Movements unterschiedlicher Player aufeinander, so tritt ein Conflict auf: Die Vires bekämpfen sich, indem die Movement-Forces der beiden Movements voneinander abgezogen werden. Nach solch einem Conflict bleibt nur das Movement über, das die stärkere Movement-Force hat, verliert aber einen Teil seiner Movement-Force durch das Aufeinandertreffen mit dem Feind. \\
Cells, auf die das Movement auf seinem Weg zur Target-Cell trifft, werden einfach ignoriert und durchlaufen. \\
Trifft ein Movement auf seine Target-Cell und gehört diese dem selben Player, der das Movement losgeschickt hat, so werden die Moving Vires des Movements als Stationed Vires in die Target-Cell aufgenommen.
Trifft ein Movement aber auf seine Target-Cell und gehört diese einem anderen Player, so tritt ebenfalls ein Conflict auf: Die Moving Vires werden von den Stationed Vires der Target-Cell abgezogen. Ist die Menge an Moving Vires größer als die Menge an Stationed Vires, so übernimmt der angreifende Player die Cell.
Überschreitet der Übergang von Moving Vires in Stationed Vires die Capacity einer Cell, so gehen die überschüssigen Moving Vires verloren.

Insgesamt gewinnt ein Player das Match, wenn er als letzter noch Cells auf dem Spielfeld besitzt oder wenn er nach 15 Minuten in der Summe die größte Cell-Force besitzt. Haben nach 15 Minuten mehrere Players insgesamt eine gleich starke Cell-Force, so gewinnt der Player, welcher die meisten Vires besitzt. Sind alle diese Werte gleich, so entsteht ein unentschieden.

\subsection{Steuerung}
\vires\ wird mit der Maus gespielt. Hält man die rechte Maustaste, so kann man die Kamera mit der Maus bewegen. Das Auf- und Abscrollen mit dem Mausrad sorgt dafür, dass die Kamera näher an das Feld heran- oder aus ihm herauszoomt . Das Klicken mit der linken Maustaste auf eine Cell, die dem Player gehört, legt die Cell als Source-Cell fest. Hält man beim Auswählen die linke Maustaste gedrückt, so können auch mehrere Cells, über die sich die Maus während dem Halten der Maustaste bewegt, als Source-Cells festgelegt werden. Nachdem die Auswahl abgeschlossen ist, kann auf eine andere Cell mit der linken Maustaste geklickt werden, um ein Movement von den ausgewählten Source-Cells zu der Target-Cell zu starten. Klickt man auf einen freien Bereich, an dem sich keine Cell befindet, so wird die Auswahl abgebrochen.
