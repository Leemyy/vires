\section{Benutzeroberfläche}
\tikzstyle{every node}=[draw=black,thick,anchor=west]
\tikzstyle{duplicate}=[draw=black,thick,dotted]
\tikzstyle{hidden}=[draw=gray!80,fill=gray!20]
\tikzstyle{hiddendupe}=[draw=gray!80,dotted,fill=gray!20]

Jeder Baum steht für ein Menü oder einen Zustand des Clients, der dem Nutzer mithilfe einer grafischen Oberfläche sichtbar gemacht wird. Insgesamt enthält der Baum alle grafischen Elemente, bei denen der Nutzer eine Aktion ausführen kann.
Alle Zweige des Baumes stehen für Funktionen, die in diesem Zustand aufgerufen werden können.
Nach dem Aufrufen der Website befindet sich der Nutzer zunächst auf der Hauptseite 
und kann von dort aus den Rest der Seite navigieren. \\
Ein gepunkteter Rand steht dabei für eine Weiterleitung in einen anderen Zustand.
Eine graue Hinterlegung zeigt an, das der Nutzer keine direkte Kontrolle über die entsprechende Funktion hat.


\vspace{0.5cm}
\begin{tikzpicture}[%
  grow via three points={one child at (0.5,-0.7) and
  two children at (0.5,-0.7) and (0.5,-1.4)},
  edge from parent path={(\tikzparentnode.south) |- (\tikzchildnode.west)}]
  \node {Hauptseite}
    child { node {Nutzernameneingabe /F0010/}}
    child{ node {Matchmaking beitreten /F0020/}
      child { node [hiddendupe]{Matchmaking}}
    }
    child [missing]{}
    child{ node {Room direkt beitreten /F0030/}
      child { node [hiddendupe]{Pre-Match-Phase}}
      child { node [hiddendupe]{Match-Phase}}
    }
  ;
\end{tikzpicture}

\vspace{0.5cm}
\begin{tikzpicture}[%
  grow via three points={one child at (0.5,-0.7) and
  two children at (0.5,-0.7) and (0.5,-1.4)},
  edge from parent path={(\tikzparentnode.south) |- (\tikzchildnode.west)}] 
  \node {Matchmaking /F0020/}
    child{ node [hiddendupe] {Pre-Match-Phase}}
    child{ node {Abbrechen}
      child { node [hiddendupe] {Hauptseite}}
    }
  ;
\end{tikzpicture}

\vspace{0.5cm}
\begin{tikzpicture}[%
  grow via three points={one child at (0.5,-0.7) and
  two children at (0.5,-0.7) and (0.5,-1.4)},
  edge from parent path={(\tikzparentnode.south) |- (\tikzchildnode.west)}]
  \node {Pre-Match-Phase /F0050/}
    child { node {Rollenwechsel /F0100/}}
    child { node {Sofortstart-Abstimmung /F0060/}
      child { node [hiddendupe] {Map-Generation}}
    }
    child [missing] {}
    child { node [hidden] {Countdown /F0060/}
      child { node [hiddendupe] {Map-Generation}}
    }
    child [missing] {}
    child { node {Zurückkehren zur Hauptseite /F0110/}
      child { node [hiddendupe] {Hauptseite}}
    }
  ;
\end{tikzpicture}

\vspace{0.5cm}
\begin{tikzpicture}[%
  grow via three points={one child at (0.5,-0.7) and
  two children at (0.5,-0.7) and (0.5,-1.4)},
  edge from parent path={(\tikzparentnode.south) |- (\tikzchildnode.west)}] 
  \node {Map-Generation /F0070/}
    child{ node [hiddendupe] {Match-Phase}}
  ;
\end{tikzpicture}

\vspace{0.5cm}
\begin{tikzpicture}[%
  grow via three points={one child at (0.5,-0.7) and
  two children at (0.5,-0.7) and (0.5,-1.4)},
  edge from parent path={(\tikzparentnode.south) |- (\tikzchildnode.west)}]
  \node {Match-Phase /F0080/}
    child { node {Zum Zuschauer wechseln /F0100/}}
    child { node {Zurückkehren zur Hauptseite /F0110/}
      child { node [hiddendupe] {Hauptseite}}
    }
    child [missing] {}
    child { node {\vires\ spielen /F0090/}
      child { node {Vires zwischen Cells bewegen}}
      child { node {Kamera bewegen}}
      child { node {Kamera zoomen}}
    }
    child [missing] {}
    child [missing] {}
    child [missing] {}
    child { node {Zuschauen /F0090/}
      child { node {Kamera bewegen}}
      child { node {Kamera zoomen}}
    }
    child [missing] {}
    child [missing] {}
    child { node [hidden] {Match-Ende}
      child { node [hiddendupe] {Pre-Match-Phase}}
    }
  ;
\end{tikzpicture}
