\section{Testszenarien}

% An \idit with a T as prefix.
% The first argument is the ID 
% of the test and the
% second argument is the name of
% the test.
\newcommand{\test}[2]{\idit{T#1}{#2}}

Alle nicht trivialen Algorithmen sollen automatisiert getestet werden, während alle trivialen Programmabläufe manuell getestet werden sollen. \\
Die folgenden Funktionen werden als nicht trivial erachtet und sollen automatisiert getestet werden:

\begin{description}
	\test{0010}{Matchmaking}
		Der Vorgang des Matchmakings soll mithilfe von automatisierten Tests getestet werden, um zu überprüfen, ob der Matchmaking-Algorithmus alle Bedingungen eines ``guten Rooms'' erfüllt.
	\test{0020}{Map-Generation}
		Der Vorgang der Map-Generation soll mithilfe von automatisierten Tests getestet werden, um zu überprüfen, ob die generierte Spielkarte alle Ansprüche erfüllt.
	\test{0030}{Serverseitige Datenverarbeitung}
		Das serverseitige Verarbeiten der empfangenen Pakete soll automatisiert getestet werden. Hiermit soll sichergestellt werden, dass der Server mit den verschiedenen Paketarten der Applikation klar kommt und die Pakete korrekt entgegengenommen werden.
	\test{0040}{Clientseitige Datenverarbeitung}
		Das clientseitige Verarbeiten der empfangenen Pakete soll automatisiert getestet werden. Hiermit soll sichergestellt werden, dass der Client mit den verschiedenen Paketarten der Applikation klar kommt und die Pakete korrekt entgegengenommen werden.
\end{description}

Alle anderen Funktionen werden manuell getestet. \\
Sollten sich bei der Entwicklung weitere Funktionen als nicht trivial und gut testbar erweisen, so werden diese ebenfalls automatisiert getestet werden.